\documentclass{ctexart}
\usepackage{graphicx}
\usepackage{amsmath}
\usepackage{amsthm}
\usepackage{amssymb}
\usepackage{fancyhdr}
\usepackage{ifthen}
\usepackage{syntonly}
\usepackage[colorlinks, CJKbookmarks=true, linkcolor=red]{hyperref}
\pagestyle{plain}
\usepackage[raggedright]{titlesec}
\newtheorem{性质}{性质}
\newtheorem{定理}{定理}
\newtheorem{推论}{推论}
\begin{document}
\title{deep learning学习笔记}
\author{计算机科学与技术系52班 杨定澄 \and 学号:2015011274 \and E-mail:892431401@qq.com}
\date{}
\maketitle
\section{前言}
虽然神经网络算法早在上个世纪80年代就提出了沿用至今的反向传播算法,但是deep learning却是在2006年才得以提出。这次作业要求阅读的“Reducing the Dimensionality of Data with Neural Networks”这篇论文,就是使深度学习算法火起来的第一步。

这篇论文涉及到许多我们之前没有学习过的概念、算法,比如PCA(主成分分析),RBM(限制性玻尔兹曼机)、autoencoder(自动编码器)等等,相关知识的缺乏给我的理解带来了一定的难度(比如以为课上讲的玻尔兹曼机就是RBM)。再查阅了一些和该论文有关的相关资料后,才稍微明白其来龙去脉。
\section{论文的主要内容}
为了便于理解,我们首先来概括一下论文的内容。

神经网络虽然早已被提出来,\textbf{但是要想得到优秀的效果其实需要初始值足够接近一个非常好的值},因此这篇论文做的第一件事,就是用一种“pretraining”的方法,对数据进行“预训练”,得到神经网络参数的一个较好的初值。这是论文中的核心部分。

接着,这篇论文究竟做了什么呢?他所做的是将数据进行降维。

在许多问题中,问题的维度是相当高的(比如图像处理问题的维度就是像素点个数那么多),直接求解会有非常多的问题。因此我们有很多数据降维的算法,比如PCA。而这篇论文最终得到的结果,就是一个新的数据降维的方法,并且通过实验表明其效果是优于PCA的。这也就是这篇论文最终的成果。

所以论文大概讲了这么几件事:

\begin{enumerate}
\item
通过“预训练”来得到一个比较好的初值(这部分直接写到了论文的最后一页)。我们称之为“pretraining”。
\item
做完pretraining后,就会输出一个“code layer”,也就是降维的结果。我们通过一个“unrolling”操作,将降维结果进行解码,得到一个与原始输入较为相似的结果。
\item
为了让解码结果与原始输入尽可能的相似,最后使用神经网络的反向传播算法,对参数进行微调。我们称之为“fine-tuning”操作。
\end{enumerate}
\section{pretraining}
这是论文最主要的内容,他主要包含两个部分。
\subsection{autoencoder}
autoencoder(自动编码器)是一个多层神经网络,但是他不是用来做分类问题的,而是学习一个输入什么就输出什么的东西。当然如果真的输入什么就输出什么,那就毫无意义了,事实上他输出的是一个用来“代表”输入数据的东西,或者说输出的是输入数据的“特征”。

首先来考虑第一个隐含层,他的维度比输入数据要小,可以看做是一个编码器。数据输入进来之后,他就会进行编码,输出一些东西来代表输入进来的数据。

但是怎么说明输出的东西代表了输入的数据呢,我们就再弄一个解码器,他通过将编码器的输出进行解码,然后要尽可能的让解码器的输出结果和最开始的输入数据一样。

通过学习神经网络的参数,我们可以让差异尽可能小。我们再让第一层编码器的输出,作为第二层编码器的输入,如此做下去。

这样子做了几层,就代表了几个原始数据的特征。每层的维度都会变小,就可以看作是将原始的输入数据进行降维了。最后一个编码器的输出,就是原始数据降维的最终结果。

autoencoder完成了pretraining所要做的事情,而让autoencoder的效果更好,可以引入RBM。
\subsection{RBM}
RBM(限制性玻尔兹曼机)不同于玻尔兹曼机里隐含层之间能互相连边,他要求可见层和隐含层构成一个二分图。即可见层内部和隐含层内部之间是没有边的。

和玻尔兹曼机相同的是,他也是假设节点是随机二值变量,概率分布满足玻尔兹曼分布。

假设可见节点集合是$v$,隐藏节点集合是$h$,一旦定好了$v$,则$h$之间就是条件独立的了;同理一旦定好了$h$,$v$之间也是条件独立的。这是由于二分图的性质。

我们仍然定义能量函数$E(v,h;\theta)=-\sum\limits_{ij}w_{ij}v_ih_j-\sum\limits_i b_i v_i-\sum\limits_j a_j h_j$,$\theta$代表所有参数(即$w_{ij},a_i,b_j$)。

而$P_{\theta}(v,h)=\frac{1}{Z}e^{-E(v,h;\theta)}$,其中$Z$是归一化常数。

我们根据隐藏节点是条件独立这一性质,可以推导出

\[P(h_j=1|v)=\frac{1}{1+exp(-\sum\limits_i w_{ij}v_i-a_j)}\]

同理可以推导出

\[P(v_i=1|h)=\frac{1}{1+exp(-\sum_j w_{ij}h_j-b_i)}\]

如果定下了$v$,就能定下$h$的概率分布,就能通过$h$的概率分布得到$v$的概率分布并进行比较。通过学习参数,可以让概率最大。

在这篇论文中,RBM和autoencoder结合了起来,他是用RBM来对autoencoder的每一层进行学习,让初值更好。
\section{unrolling}
再pretraining的时候既学习编码器的参数,同时也学了解码器的参数。

由于所有的解码器参数已知,解码也就非常容易了,直接推回去就好。
\section{fine-tuning}
最后再算一次梯度下降来让差异函数最小。

他这里的差异函数不是用距离平方和进行度量,而是用了一种叫交叉熵损失函数的度量方法,最小化$-\sum\limits_i p_i \log \hat{p_i} - \sum\limits_i (1-p_i) \log(1-\hat{p_i})$


\end{document}
